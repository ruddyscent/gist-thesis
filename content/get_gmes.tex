% -*- coding: utf-8 -*-
%
% Kyungwon Chun <ruddyscent@gmail.com>
%

\section{How to get GMES}
GMES is owned by GIST\footnote{\url{www.gist.ac.kr}} and licensed under GPLv3\footnote{\url{http://www.gnu.org/licenses/quick-guide-gplv3.html}}. You can download the distribution file from GMES project page\footnote{\url{http://sourceforge.net/projects/gmes}}. GMES project provides a mailing-list, \emph{gmes-user} and forums. General questions about using gmes are recommended to use the mailing-list. For the other questions, feel free to contact the authors\footnote{\href{mailto:ruddyscent@gmail.com}{ruddyscent@gmail.com}}. Any contribution are welcomed.

\section{Installation and use of GMES}
You can find the installation instruction at the README file in GMES distribution. Though GMES tries to support various platform, the recent version of GMES only test on Ubuntu/Linux. All of the prerequisites of GMES are provided in Debian/Linux and Ubuntu/Linux distributions.

The current version of GMES, does not have a separate user manual. Please, refer this thesis and built-in Python help system.

\section{The file map of GMES distribution}
The list up of files in GMES distribution in this section is to deliver a information to the developer who wants to enhance a particular function. He/She may find an idea of where in the code each function is carried out, then go straight to the relevant part of the code.

\begin{itemize}
\item COPYING: license agreement
\item examples: Examples of the use of the GMES package, see README for details.
\item gmes/: Top level package
  \begin{itemize}
    \item fdtd.py: fdtd module which provide various simulation classes suitable for 1, 2, and 3 dimensions.
    \item file\_io.py: file\_io module which provides classes related to the file I/O.
    \item geometry.py: geometry module which provide coordinate and geometric primitives.
    \item \_\_init\_\_.py: initializes the GMES package.
    \item pw\_source.py: pw\_source module about source update mechanism
    \item show.py: show module which provides classes for display of EM fields and geometrical structures.
    \item source.py: source module to define the input sources
  \end{itemize}
\item README: Compile \& installation instructions
\item setup.py: Setup script using the Distutils
\item src/: C\verb!++! \& Cython source directory
  \begin{itemize}
    \item constant.cc: C\verb!++! source code of the constant module which defines physical and simulation constants
    \item constant.hh: C\verb!++! header of the constant module
    \item constant.i: SWIG interface file for the constant module
    \item material.pyx: Cython source of the material module which defines the propagating medium
    \item numpy.i: SWIG interface file for NumPy arrays
    \item pw\_const.cc: C\verb!++! source code for the update mechanism of `Const' medium
    \item pw\_const.hh: C\verb!++! header for the update mechanism of `Const' medium
    \item pw\_cpml.cc: C\verb!++! source code for the update mechanism of `Cpml' medium
    \item pw\_cpml.hh: C\verb!++! header for the update mechanism of `Cpml' medium
    \item pw\_dcp.cc: C\verb!++! source code for the update mechanism of `Dcp' medium
    \item pw\_dcp.hh: C\verb!++! header for the update mechanism of `Dcp' medium
    \item pw\_dielectric.cc: C\verb!++! source code for the update mechanism of 'Dielectric' medium
    \item pw\_dielectric.hh: C\verb!++! header for the update mechanism of `Dielectric' medium
    \item pw\_drude.cc: C\verb!++! source code for the update mechanism of `Drude' medium
    \item pw\_drude.hh: C\verb!++! header for the update mechanism of `Drude' medium
    \item pw\_dummy.cc: C\verb!++! source code for the update mechanism of `Dummy' medium
    \item pw\_dummy.hh: C\verb!++! header for the update mechanism of `Dummy' medium
    \item pw\_lorentz.cc: C\verb!++! source for the update mechanism of `Lorentz' medium
    \item pw\_lorentz.hh: C\verb!++! header for the update mechanism of `Lorentz' medium
    \item pw\_material.cc: C\verb!++! source of the base template class for `PwMaterial' type
    \item pw\_material.hh: C\verb!++! header of the base template class for `PwMaterial' type
    \item pw\_material.i: A SWIG interface file for the pw\_material module
    \item pw\_upml.cc: C\verb!++! source for the update mechanism of `Cpml' medium
    \item pw\_upml.hh: C\verb!++! header for the update mechanism of `Cpml' medium
    \item pygeom.pyx: A Cython source code of the pygeom module which provides the geometric primitives
  \end{itemize}
\item tests/: unit testing files
  \begin{itemize}
    \item pw\_const\_test.py: A testing file for classes in the pw\_const module
      \item pw\_cpml\_test.py: A testing file for classes in the pw\_cpml module
      \item pw\_dcp\_test.py: A testing file for classes in the pw\_dcp module
      \item pw\_dielectric\_test.py: A testing file for classes in the pw\_dielectric module
      \item pw\_drude\_test.py: A testing file for classes in the pw\_drude module
      \item pw\_dummy\_test.py: A testing file for classes in the pw\_dummy module
      \item pw\_lorentz\_test.py: A testing file for classes in the pw\_lorentz module
      \item pw\_upml\_test.py: A testing file for classes in the pw\_upml module
  \end{itemize}
\item utils/: miscellaneous utility files
  \begin{itemize}
  \item combine.py: Combine the divided npy files.
  \item dia\_util.py: A Dia module to generate a class diagram for GMES.
  \end{itemize}
\item VERSION: Indicates the current version number.
\end{itemize}
