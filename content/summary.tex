% -*- coding: utf-8 -*-

We have described in this paper about GMES, a Python package which solves Maxwell's equations using the FDTD method. The design of GMES uses a complete OOP approach, thereby providing an FDTD code that is well structured and easy to understand for the users without compromising the simplicity and speed of FDTD algorithm.

Instead of a conventional update scheme where the voxels are updated in a sequential manner, GMES has adopted a unique strategy where the voxels are grouped and then updated according to its material type. This design feature could minimize the virtual calls as well as avoid the usage of conditionals, thereby maintaining the simplicity and speed of FDTD implementation. 

GMES is a free package and allows the users to add new features, without complicating the FDTD algorithm. It also  provides ample scope for the users to experiment and implement new physical phenomena and design problems related to electromagnetics. Various kinds of sources and boundary conditions can be handled and implemented in GMES with ease. We anticipate that these features would help GMES to find its acceptance among researchers who want to make new FDTD codes without sacrificing the simplicity and speed of FDTD.

Also, we have shown the implementation of DCP model for describing dispersive media using PLRC and ADE schemes into FDTD algorithm. It was shown that DCP model can correctly describe the experimentally reported permittivity values of various metals. Comparison of PLRC and ADE implementations showed that PLRC scheme requires lesser memory, while ADE scheme requires only fewer arithmetic operations. It was found that the numerical error of the PLRC implementation was less compared to that of ADE implementation for the metals considered. Stability analysis of both schemes showed that ADE scheme can have the same stability condition to that of the non-dispersive FDTD case. Finally, the implementation schemes were applied in studying the transmittance and reflectance of thin metal films, and excellent agreement was observed between the analytical and numerical results, thus validating our implementations.
